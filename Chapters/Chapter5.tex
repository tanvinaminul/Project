% Chapter 5

\chapter{Conclusion and Implication for Future} % Main chapter title

\label{Chapter5} % For referencing the chapter elsewhere, use \ref{Chapter3} 

\lhead{Chapter 5. \emph{Conclusion and Implication for Future}} % This is for the header on each page - perhaps a shortened title

%----------------------------------------------------------------------------------------
\section{Summary of the Study}

This simulation is based on blackhole attack detection in a vanet when aodv is using as a routing protocol. Vanet is used in the vehicle so it has a huge impact in real life. And many network-level attackers prefer blackhole attack because it maintains communication between legitimate nodes and drops packet. 

\section{Conclusions}

Despite a large number of vehicles in this simulation, this process of detection blackhole is proper and the process that we use is so far won't used yet. The parameters of the simulation give a clear idea about the blackhole. Considering three parameters helps to decide the accuracy of this work. 

\section{Recommendations}
Vanet is a new technology in traffic maintenance and driving support system. Much work is done in the field of vanet nowadays. they are maximum on the security issue. Mainly in the aodv protocol.\\
P. Anand Babu designed attack detection in vanet. Information about the attacker is stored in the knowledge base. 


\section{Implication for Further Study}

Though blackhole detection is efficient it can't detect any malicious node with its previous knowledge. No machine learning is used here. Using machine learning make this an uncomputable detection tool.\\
Simulation has occurred 8 minutes and 50 nodes. If this simulation is run for more time with a more nodes the result will be more accurate and proper.\\
In this work, simulation is tested only in aodv protocol. In the future, another protocol may be implemented.
